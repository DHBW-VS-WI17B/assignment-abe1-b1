In diesem Kapitel werden die zu erfüllenden Eigenschaften und Leistungen an das System gesammelt. Dadurch werden strukturierte und konkrete Anforderungen an das Gesamtsystem bereitgestellt.
Da die Anforderungen ausschließlich der Aufgabenstellung entstammen, findet keine Gewichtung der einzelnen Anforderungen hinsichtlich der Umsetzungsrelevanz statt.

\autoref{tab:general_requirements} beinhaltet die allgemeinen Anforderungen an das Gesamtsystem.

\newcounter{tablenumber}
\newcommand\inctablenumber{\stepcounter{tablenumber}}
\newcommand\tablenumber{\arabic{tablenumber}}
\newcounter{rownumber}
\newcommand\incrownumber{\stepcounter{rownumber}}
\newcommand\rownumber{\arabic{rownumber}}
\setcounter{tablenumber}{1}
\setcounter{rownumber}{1}
\begin{dhbwtable}{caption={Allgemeine Anforderungen},label=tab:general_requirements,float=H}
    \begin{tabularx}{\textwidth}{l | X}
        \toprule
        \textbf{ID} & \textbf{Beschreibung}  \\\midrule
        \tablenumber .\rownumber & Kunde und Verwalter sollen über Clients auf die Verwaltungskomponente zugreifen können. \\\midrule\incrownumber
        \tablenumber .\rownumber & Die Anzahl der verfügbaren Tickets einer Veranstaltung soll begrenzt sein. \\\midrule\incrownumber
        \tablenumber .\rownumber & Die Anzahl der verkauften Tickets einer Veranstaltung pro Kunde soll begrenzt sein. \\\midrule\incrownumber
        \tablenumber .\rownumber & Der Verkauf von Tickets soll zu einem bestimmten Zeitpunkt starten und nur innerhalb eines bestimmten Zeitraums möglich sein. \\\midrule\incrownumber
        \tablenumber .\rownumber & Der Ticketverkauf soll zentral verwaltet werden. \\\midrule\incrownumber
        \tablenumber .\rownumber & Mehrere Kunden sollen gleichzeitig auf den Ticketverkauf zugreifen können. \\\midrule\incrownumber
        \tablenumber .\rownumber & Es sollen folgende Daten zu einem Kunden verwaltet werden: Name, Adresse, erworbene Tickets mit Kaufdatum, Jahresbudget. \\\midrule\incrownumber
        \tablenumber .\rownumber & Es sollen folgende Daten zu einer Veranstaltung verwaltet werden: ID, Name, Datum, Ort, Preis eines Tickets, Anzahl an Tickets, erlaubte Anzahl an Tickets pro Kunde. \\\midrule\incrownumber
        \tablenumber .\rownumber & Das Budget des Kunden soll veranstaltungsübergreifend gelten. \\\midrule\incrownumber
        \tablenumber .\rownumber & Das Budget eines Kunden darf nur dem Kunden selbst bekannt sein.  \\\bottomrule
    \end{tabularx}    
\end{dhbwtable}

\autoref{tab:customer_client_requirements} beinhaltet die Anforderungen an den Client des Kunden.

\setcounter{tablenumber}{2}
\setcounter{rownumber}{1}
\begin{dhbwtable}{caption={Anforderungen an den Client des Kunden},label=tab:customer_client_requirements,float=H}
    \begin{tabularx}{\textwidth}{l | X}
        \toprule
        \textbf{ID} & \textbf{Beschreibung}  \\\midrule
        \tablenumber .\rownumber & Der Kunde soll eine Liste mit Namen und IDs der angebotenen Veranstaltungen abrufen können. \\\midrule\incrownumber
        \tablenumber .\rownumber & Der Kunde soll die Daten einer bestimmten Veranstaltung abrufen können. \\\midrule\incrownumber
        \tablenumber .\rownumber & Der Kunde soll in einem Aufruf mehrere Tickets für eine Veranstaltung erwerben können. \\\midrule\incrownumber
        \tablenumber .\rownumber & Der Kunde soll die Daten zu den von ihm bestellten Tickets abrufen können. Dabei soll entweder nach Bestelldatum oder Zeitpunkt der Veranstaltung gefiltert werden können. \\\midrule\incrownumber
        \tablenumber .\rownumber & Der Kunde soll den aktuellen Stand seines Budgets prüfen können. \\\bottomrule
    \end{tabularx}    
\end{dhbwtable}

\autoref{tab:admin_client_requirements} beinhaltet die Anforderungen an den Client des Verwalters.

\setcounter{tablenumber}{3}
\setcounter{rownumber}{1}
\begin{dhbwtable}{caption={Anforderungen an den Client des Verwalters},label=tab:admin_client_requirements,float=H}
    \begin{tabularx}{\textwidth}{l | X}
        \toprule
        \textbf{ID} & \textbf{Beschreibung}  \\\midrule
        \tablenumber .\rownumber & Der Verwalter soll eine Veranstaltung anlegen können. \\\midrule\incrownumber
        \tablenumber .\rownumber & Der Verwalter soll eine Liste mit Namen und IDs der angebotenen Veranstaltungen abrufen können. \\\midrule\incrownumber
        \tablenumber .\rownumber & Der Verwalter soll die Daten einer bestimmten Veranstaltung abrufen können. \\\midrule\incrownumber
        \tablenumber .\rownumber & Der Verwalter soll die Anzahl der verkauften Tickets pro Veranstaltung abrufen können. \\\bottomrule
    \end{tabularx}    
\end{dhbwtable}
