In diesem Kapitel werden die zu erfüllenden Eigenschaften und Leistungen an das System gesammelt. Dadurch werden strukturierte und konkrete Anforderungen an das Gesamtsystem bereitgestellt.
Nachfolgend wird zwischen funktionalen und nicht-funktionalen Anforderungen unterschieden.
Da die Anforderungen ausschließlich der Aufgabenstellung entstammen, findet keine Gewichtung der einzelnen Anforderungen hinsichtlich der Umsetzungsrelevanz statt.

\subsubsection{Funktionale Anforderungen}\label{sec:functional_requirements}

Funktionale Anforderungen beschreiben die gewünschten spezifischen Funktionalitäten.\ifootcite[37]{grande2011}
Tabelle X beinhaltet die funktionalen Anforderungen an das Gesamtsystem.

\subsubsection{Nicht-funktionale Anforderungen}\label{sec:non_functional_requirements}

Nicht-funktionale Anforderungen sind meist unspezifisch und bestimmen Qualitätsmerkmale und Randbedingungen.\ifootcite[39]{grande2011}
Tabelle Y beinhaltet die nicht-funktionalen Anforderungen an das Gesamtsystem.
