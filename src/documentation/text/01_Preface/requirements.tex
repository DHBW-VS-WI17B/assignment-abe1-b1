Anforderungen legen den Grundstein für jede zu entwickelnde Anwendung und \dfootcite[5]{grande2011}{[...] beschreiben Eigenschaften, Funktionalitäten und Qualitäten, die ein Produkt bekommen soll}.
Präzise ausformulierte und klar definierte Anforderungen stellen sicher, dass der Kunde seine gewünschte Anwendung erhält.\ifootcite[6]{grande2011} 
Um dies zu Gewährleisten sollten Anforderungen in schriftlicher oder elektronischer Form dokumentiert werden.
Zudem dienen die  Anforderungen als Grundlage für weitere Aktivitäten und Implementierungsschritte, wie bspw. Tests, bis zur finalen Anwendung. \ifootcite[6 - 7]{grande2011} 

Grande beschreibt das Fünfeck der Anforderungs-Pflicht-Attribute, welches sicherstellt, dass alle relevanten Anforderungskriterien benannt werden. 
Die Bestandteile des Fünfecks \dfootcite[43]{grande2011}{[…] sind die Identifikation, der Name, die Beschreibung, der Status und die Version der Anforderung}.
Da die zu entwickelnde Anwendung nur für eine einmalige Verwendung konzipiert wird, sind die beiden Pflicht-Attribute Status und Version nicht notwendig. 
Darüber hinaus werden die im System zu verwalteten Daten im nachfolgenden Kapitel 2 berücksichtigt.

Ein Softwaresystem stellt mit seinen Systemkomponenten Funktionalitäten für Benutzer oder andere Systeme bereit. 
Funktionalität beinhalten alle Funktionen des Systems oder einer Systemkomponente.\ifootcite[127]{balzert2009}
Systemanforderungen lassen sich in funktionale und nicht-funktionale Anforderungen unterteilen, welche im Folgenden beschrieben werden.

