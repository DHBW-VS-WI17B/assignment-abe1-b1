Die vorliegende Seminararbeit thematisiert den Anwendungsfall eines Ticketverkaufs für Veranstaltungen.
Kunden und Verwalter erhalten über spezielle Clients Zugriff auf eine Verwaltungskomponente, dessen Implementierung auf einem Aktorensystem basiert.
Hierbei muss die Implementierung des Systems alle Anwendungsfälle erfüllen.
Dafür müssen zunächst die Anforderungen an das System gesammelt werden, bevor eine Unterteilung in funktionale und nicht-funktionale Anforderungen stattfindet.
Darauf aufbauend wird sowohl die Gesamtarchitektur als auch die Architektur der Teilsysteme entworfen.
Sowohl die Gesamtarchitektur des Systems als auch die Programmiersprachen und Bibliotheken dürfen frei gewählt werden.
