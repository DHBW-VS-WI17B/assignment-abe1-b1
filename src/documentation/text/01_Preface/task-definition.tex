% Aufgabenstellung

Die vorliegende Seminararbeit bearbeitet den Anwendungsfall eines Ticketverkaufs. 
Dieser soll mit einem Aktorensystem, welches im Laufe der Arbeit erläutert wird, umgesetzt werden.
Das Hauptaugenmerkt des Anwendungsfalls besteht darin, dass der Ticketverkauf zentral verwaltet werden soll, um mehreren Kunden gleichzeitig den Ticketverkauf zu ermöglichen.
Jede der angebotenen Veranstaltungen besitzt ein vorgegebenes Ticketkontingent, welche jeweils ab einem festgelegtem Zeitpunkt und für einen festgelegten Zeitraum erworben werden können.
Die Kunden des Ticketverkaufs besitzen ein maximales Jahresbudget für den Ticketkauf aller Veranstaltungen. Zudem verfügen sie nur über ein limitiertes Ticketkontingent pro Veranstaltung.

Für die zu verwendende Programmiersprache und das Aktorensystem gibt es keine Vorgaben.
Der betreuende Dozent, in dieser Arbeit Kunde genannt, empfiehlt jedoch die Verwendung des Aktorensystems \enquote{Thespian}.
Der zu implementierende Ticketverkauf kann über ein Kommandozeilen-Interface für Clients und ohne Benutzeroberfläche gestaltet werden.
Die Clients und Verwaltungskomponenten können gemeinsam innerhalb eines Prozesses auf einem Rechner ablaufen.
Die zu entwickelnde Anwendung benötigt nicht zwingend eine verteilte Lauffähigkeit.

% Aufbau der Arbeit?
