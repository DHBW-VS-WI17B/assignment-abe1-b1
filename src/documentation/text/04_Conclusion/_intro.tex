Ziel der Seminararbeit war es, ein Aktorensystem für den Ticketverkauf von Veranstaltungen zu konzipieren und zu implementieren. 
Die Grundlage für die Konzeption waren die, in Kapitel 1 und 2 dargestellten Anforderungen und Anwendungsfälle. 

Das Aktorensystem wurde durch eine Client-Server-Architektur realisiert, welche serverseitig das Aktorensystem Thepsian und einen Webserver über eine REST-API zur Verfügung stellt. 
Die Interaktion zwischen Client und Server basiert auf HTTP-Anfragen, die über eine REST-API an das Aktorensystem übermittelt werden. 
Zuletzt verarbeitet die integrierte SQLite-Datenbank die gesendeten Anfragen. 
Das verwendete Datenmodell bestehend aus den Entitäten Event, Customer und Ticket soll die Funktionalität des Aktorensystems gewährleisten.

Die Gesamtarchitektur stellt für beide Teilsysteme jeweils eine CLI für die Interaktion zur Verfügung. 
Die drei Aktorenklassen DBActor, EventsActor und CustomersActor wurden für die Implementierung des Ticketverkaufs eingesetzt. 
Der Nachrichtenaustausch zwischen den Aktoren erfolgt über die Klasse ActorMessage, die abschließend in Kapitel 3.4 beschrieben wurde.