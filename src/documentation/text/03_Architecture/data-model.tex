Das Datenmodell setzt sich aus drei Entitäten \textit{Event}, \textit{Customer} und \textit{Ticket} zusammen.
\autoref{fig:uml_data_model} beinhaltet eine Abbildung des Datenmodells als Entity-Relationship-Modell.

\begin{dhbwfigure}{caption=Datenmodell,label=fig:uml_data_model,source={Eigene Darstellung.},float=H}
    \begin{plantuml}
        @startuml

        scale 0.7
        skinparam linetype ortho
        left to right direction

        entity "Event" as event {
            * id: integer
            --
            * name: string
            * date: integer
            * location: string
            * ticket_price: integer
            * max_tickets: integer
            * max_tickets_per_customer: integer
            * sale_start_date: integer
            * sale_period: integer
        }   
        entity "Ticket" as ticket {
            * id: integer
            --
            * order_date: integer
            --
            * customer_id: integer
            * event_id: integer
        }
        entity "Customer" as customer {
            * id: integer
            --
            * name: string
            * budget: integer
            * address: string
        }

        event ||--o{ ticket
        ticket }o--|| customer

        @enduml
    \end{plantuml}
\end{dhbwfigure}

