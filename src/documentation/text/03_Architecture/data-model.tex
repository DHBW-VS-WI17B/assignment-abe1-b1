Das Datenmodell setzt sich aus den drei Entitäten \textit{Event}, \textit{Customer} und \textit{Ticket} zusammen.
\autoref{fig:uml_data_model} beinhaltet eine Abbildung des Datenmodells als Entity-Relationship-Modell.

\begin{dhbwfigure}{caption=Datenmodell,label=fig:uml_data_model,source={Eigene Darstellung.},float=H}
    \begin{plantuml}
        @startuml

        scale 0.7
        skinparam linetype ortho
        left to right direction

        entity "Event" as event {
            * id: integer
            --
            * name: string
            * date: integer
            * location: string
            * ticket_price: integer
            * max_tickets: integer
            * max_tickets_per_customer: integer
            * sale_start_date: integer
            * sale_period: integer
        }   
        entity "Ticket" as ticket {
            * id: integer
            --
            * order_date: integer
            --
            * customer_id: integer
            * event_id: integer
        }
        entity "Customer" as customer {
            * id: integer
            --
            * name: string
            * budget: integer
            * address: string
        }

        event ||--o{ ticket
        ticket }o--|| customer

        @enduml
    \end{plantuml}
\end{dhbwfigure}\unskip

Die Entität \textit{Ticket} dient grundsätzlich der Realisierung der n:m-Beziehung zwischen \textit{Event} und \textit{Customer}, fügt dieser Beziehung jedoch auch eine weitere Eigenschaft, das Kaufdatum, hinzu.
Das Jahresbudget des Kunden wird einmalig gesetzt und anschließend bei Bedarf für das jeweilige Jahr ausgehend von diesem Wert und anhand der gekauften Tickets und Ticketpreise berechnet.
Somit müssen nicht mehrere Budgetwerte eines Kunden gespeichert werden. 
Zu diesen kommt es, wenn der Kunde Tickets von Events kauft, die in unterschiedlichen Jahren stattfinden.
