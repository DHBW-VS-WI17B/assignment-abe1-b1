Für den Nachrichtenaustausch zwischen den Aktoren wird die in \autoref{fig:uml_actor_system_class_diagramm} abgebildete Klasse \textit{ActorMessage} verwendet.

\begin{dhbwfigure}{caption=Klassen im Aktorensystems,label=fig:uml_actor_system_class_diagramm,source={Eigene Darstellung.},float=H}
    \begin{plantuml}
        @startuml

        class ActorMessage {
            action: String
            payload: Object 
            customer_id: Integer
            response_to: Object
            error: ActorMessageError
            ActorMessage()
        }

        class ActorMessageError {
            message: String
            http_code: Integer 
            ActorMessageError()
        }

        enum EventsActorAction {
            EVENTS_LIST
            EVENTS_ADD
            EVENTS_GET
            EVENTS_SALES
            EVENTS_PURCHASE
        }

        enum CustomersActorAction {
            CUSTOMERS_ADD
            CUSTOMERS_BUDGET
            CUSTOMERS_TICKETS
        }

        ActorMessage --> ActorMessageError
        ActorMessage --> EventsActorAction
        ActorMessage --> CustomersActorAction

        @enduml
    \end{plantuml}
\end{dhbwfigure}\unskip

Die möglichen Anwendungsfälle werden durch die Klassen \textit{EventsActorAction} und \textit{CustomersActorAction} abgebildet.
Da es sich bei den Aktionen \textit{EVENTS\_ADD}, \textit{CUSTOMERS\_ADD} und \textit{EVENTS\_PURCHASE} um Aktionen handelt, bei denen Schreibzugriffe auf die Datenbank durchgeführt werden müssen, werden Nachrichten mit diesen Aktionen an den globalen DbActor weitergeleitet.
Die Eigenschaft \textit{payload} enthält die eigentliche Nachricht an den Zielaktor.
Stammt die eingehende Anfrage von einem Kunden besitzt das Attribut \textit{customer\_id} die Kunden-ID des jeweiligen Kunden.
Die Eigenschaft \textit{response\_to} enthält die Adresse des Aktors, von dem die Nachricht ursprünglich stammt und an den somit auch die Antwort geschickt werden sollte.
Im Fehlerfall wird ein Objekt der Klasse \textit{ActorMessageError} der Eigenschaft \textit{error} zugewiesen.
