Das Aktorensystem setzt sich aus den folgenden drei Aktorenklassen zusammen:
\begin{enumerate}
    \item \textbf{DbActor}: Aktoren dieser Klasse sind für die Kommunikation mit der Datenbank zuständig.
    Es existiert genau ein Aktor dieser Klasse, der ausschließlich für Schreibzugriffe verwendet wird.
    Dieser Aktor besitzt einen vordefinierten globalen Namen.
    Der globale Name identifiziert den Aktor eindeutig und ermöglicht anderen Aktoren die Referenzierung dieser Instanz.\ifootcite{thespianpy_com_guide}
    Dadurch wird sichergestellt, dass alle Schreibzugriffe gemäß der Aufgabenstellung sequentiell ablaufen.
    Bei Lesezugriffen ist pro Zugriff ein eigener Aktor verantwortlich.
    Dies ermöglicht eine nebenläufige Verarbeitung aller eingehenden Anfragen dieser Art.
    \item \textbf{EventsActor}: Ein Aktor dieser Klasse ist für die Verarbeitung von Aktionen, die sich auf Veranstaltungen beziehen, zuständig.
    Dazu gehört u.\ a.\ der Kauf von Tickets.
    Der Webserver erstellt mit jeder eingehende Anfrage auf die Basisroute \enquote{/api/events} eine Instanz dieser Aktorenklasse. 
    Die Aufgabe des Aktors ist anschließend die Weiterleitung der benötigten Informationen an die korrekte \textit{DbActor}-Instanz.
    \item \textbf{CustomersActor}: Ein Aktor dieser Klasse ist für die Verarbeitung von Aktionen, die sich auf Kunden beziehen, zuständig.
    Dazu gehört u.\ a.\ das Abfragen des Budgets für ein ausgewähltes Jahr durch den Kunden.
    Der Webserver erstellt mit jeder eingehende Anfrage auf die Basisroute \enquote{/api/customers} eine Instanz dieser Aktorenklasse. 
    Die Aufgabe des Aktors ist anschließend die Weiterleitung der benötigten Informationen an die korrekte \textit{DbActor}-Instanz.
\end{enumerate}

\begin{dhbwfigure}{caption=Aktorensystem,label=fig:uml_overall_architecture,source={Eigene Darstellung.},float=H}
    \begin{plantuml}
        @startuml

            skinparam componentStyle uml2
            left to right direction

            component "Webserver" as WebServer
            database "Datenbank" as Database

            package "Aktorensystem" {
                component "EventsActor" as EventsActor
                component "CustomersActor" as CustomersActor
                component "DbActor" as DbActor

                EventsActor <--> DbActor
                CustomersActor <--> DbActor
            }

            WebServer --> EventsActor
            WebServer --> CustomersActor
            DbActor --> Database

        @enduml
    \end{plantuml}
\end{dhbwfigure}\unskip

\autoref{fig:uml_overall_architecture} veranschaulicht die Architektur des Aktorensystems, das in \autoref{fig:uml_overall_architecture} bereits in die Gesamtarchitektur eingeordnet wurde.
