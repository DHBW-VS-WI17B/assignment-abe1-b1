Das Aktorensystem setzt sich aus den folgenden drei Aktorenklassen zusammen:
\begin{enumerate}
    \item \textbf{EventsActor}: Der EventsActor ... Von ihm existieren keine globalen Instanzen. Der EventsActor ist dafür zuständig eingehende Nachrichten an die korrekte DbActor-Instanz weiterzuleiten.
    \item \textbf{CustomersActor}: Der CustomersActor ... Von ihm existieren keine globalen Instanzen. Der CustomersActor ist dafür zuständig eingehende Nachrichten an die korrekte DbActor-Instanz weiterzuleiten.
    \item \textbf{DbActor}: Der DbActor ... Von ihm existieren sowohl globale als auch nicht-globale Instanzen.
\end{enumerate}

\autoref{fig:uml_overall_architecture} veranschaulicht die Architektur des Aktorensystems, das in \autoref{fig:uml_overall_architecture} bereits in die Gesamtarchitektur eingeordnet wurde.

\begin{dhbwfigure}{caption=Aktorensystem,label=fig:uml_overall_architecture,source={Eigene Darstellung.},float=H}
    \begin{plantuml}
        @startuml

            skinparam componentStyle uml2
            left to right direction

            component "Webserver" as WebServer
            database "Datenbank" as Database

            package "Aktorensystem" {
                component "EventsActor" as EventsActor
                component "CustomersActor" as CustomersActor
                component "DbActor" as DbActor

                EventsActor <--> DbActor
                CustomersActor <--> DbActor
            }

            WebServer --> EventsActor
            WebServer --> CustomersActor
            DbActor --> Database

        @enduml
    \end{plantuml}
\end{dhbwfigure}\unskip

Jede eingehende Anfrage eines Clients an den Webserver erzeugt einen neuen EventsActor bzw.\ CustomersActor.
Diese leiten die Anfrage im Falle eines Lesezugriffs an einen globalen DbActor weiter, welcher ausschließlich zur Serialisierung der Schreibzugriffe auf die Datenbank dient.
Erfolgt hingegen ein Lesezugriff, erzeugt der jeweilige EventsActor bzw.\ CustomersActor ausschließlich für diese Operation einen neuen DbActor, welcher parallel zu anderen Aktoren lesend auf die Datenbank zugreifen kann.
