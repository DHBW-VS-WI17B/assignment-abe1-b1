Die Gesamtarchitektur beschreibt die grundlegenden Komponenten und deren Zusammenspiel.
Zum Einsatz kommt eine Client-Server-Architektur.
Beide Teilsysteme besitzen jeweils eine \ac{CLI}, über die Benutzer mit der jeweiligen Komponente interagieren können.
Die Eingaben müssen hierbei einer bestimmten Syntax folgenden.
Informationen zu dieser Syntax erhält ein Benutzer durch Aufrufen der \ac{CLI} mit dem \enquote{help}-Parameter.

\begin{dhbwfigure}{caption=Gesamtarchitektur,label=fig:uml_overall_rchitecture,source={Eigene Darstellung.},float=H}
    \begin{plantuml}
        @startuml

            skinparam componentStyle uml2
            left to right direction

            package "Client-CLI" {
                component "Client" as Client
            }

            package "Server-CLI" {
                component "Webserver" as WebServer
                component "Aktorensystem" as ActorSystem
                database "Datenbank" as Database

                WebServer --> ActorSystem
                Database <- ActorSystem
            }

            Client --> WebServer: HTTP

        @enduml
    \end{plantuml}
\end{dhbwfigure}

\autoref{fig:uml_overall_rchitecture} bildet die Gesamtarchitektur des Systems ab.
Client und Server kommunizieren über \ac{HTTP} miteinander.
Hierfür implementiert der Server neben dem Aktorensystem \enquote{Thespian} einen Webserver und stellt eine \ac{API} zur Verfügung. % TODO: Link
Der Client kann \ac{HTTP}-Anfragen an den Webserver senden und so über die \ac{API} mit dem Aktorensystem kommunizieren, welches wiederum mit der SQLite-Datenbank interagiert. % TODO: Link

% TODO: Tabelle mit Informationen zur API einfügen
