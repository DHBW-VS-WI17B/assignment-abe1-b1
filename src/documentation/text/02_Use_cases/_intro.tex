Im Rahmen dieses Kapitels werden die funktionalen Anforderungen an das System in Form von Anwendungsfällen beschrieben. 

\begin{dhbwfigure}{caption=Anwendungsfall-Diagramm,label=fig:use_case_diagram,float=H}
    \begin{plantuml}
        @startuml

            left to right direction
            skinparam packageStyle rectangle

            :Kunde: as Customer
            :Verwalter: as Admin

            rectangle Ticketverkauf {
                Customer --- (Tickets kaufen)
                Customer --- (Budget abrufen)
                Customer --- (Bestellte Tickets abrufen)
                (Bestellte Tickets filtern) .> (Bestellte Tickets abrufen) : extends
                Customer ---- (Veranstaltungsliste abrufen)
                Customer ---- (Veranstaltungsdaten abrufen)

                (Verkaufte Tickets abrufen) --- Admin
                (Event erstellen) --- Admin
                (Veranstaltungsliste abrufen) ---- Admin
                (Veranstaltungsdaten abrufen) ---- Admin
            }
        @enduml
    \end{plantuml}
\end{dhbwfigure}

\autoref{fig:use_case_diagram} beinhaltet ein Anwendungsfalldiagramm.
Es dient der Beschreibung von elementaren Funktionen und \dfootcite[59]{objektorientierte_systemanalyse}{... stellt das Zusammenspiel der Anwendungsfälle eines Systems untereinander und mit den Akteuren dar}.

Die ermittelten Anwendungsfälle werden in den Tabellen TODO bis TODO genauer erläutert. % TODO

\begin{dhbwtable}{caption={Anwendungsfall: Veranstaltungsliste abrufen},label=tab:use_case_list_events,float=H}
    \begin{tabularx}{\textwidth}{lX}
        \toprule
        \textbf{Merkmal} & \textbf{Beschreibung}  \\ \midrule
        Beschreibung    & Es kann eine Liste mit den Namen und IDs der angebotenen Veranstaltungen abgerufen werden. \\
        Akteure         & Kunde, Verwalter \\
        Vorbedingungen  & Keine \\
        Ablauf          & \begin{enumerate}
            \item Der Akteur stellt über den Client eine Anfrage an die Verwaltungskomponente.
            \item Der Client gibt eine List mit Veranstaltungen aus.
        \end{enumerate} \\\bottomrule
    \end{tabularx}    
\end{dhbwtable}

\begin{dhbwtable}{caption={Anwendungsfall: Veranstaltungsdaten abrufen},label=tab:use_case_get_event_data,float=H}
    \begin{tabularx}{\textwidth}{lX}
        \toprule
        \textbf{Merkmal} & \textbf{Beschreibung}  \\ \midrule
        Beschreibung    & Zu einer bestimmten angebotenen Veranstaltung kann der Namen und die ID abgerufen werden. \\
        Akteure         & Kunde, Verwalter \\
        Vorbedingungen  & Keine \\
        Ablauf          & \begin{enumerate}
            \item Der Akteur stellt über den Client eine Anfrage an die Verwaltungskomponente.
            \item Der Client gibt entweder die Daten zu der Veranstaltung oder den Hinweis, das die Veranstaltung nicht gefunden werden konnte, aus.
        \end{enumerate} \\\bottomrule
    \end{tabularx}    
\end{dhbwtable}

\begin{dhbwtable}{caption={Anwendungsfall: },label=tab:use_case_,float=H}
    \begin{tabularx}{\textwidth}{lX}
        \toprule
        \textbf{Merkmal} & \textbf{Beschreibung}  \\ \midrule
        Beschreibung    &  \\
        Akteure         &  \\
        Vorbedingungen  &  \\
        Ablauf          & \begin{enumerate}
            \item 
        \end{enumerate} \\\bottomrule
    \end{tabularx}    
\end{dhbwtable}

\begin{dhbwtable}{caption={Anwendungsfall: },label=tab:use_case_,float=H}
    \begin{tabularx}{\textwidth}{lX}
        \toprule
        \textbf{Merkmal} & \textbf{Beschreibung}  \\ \midrule
        Beschreibung    &  \\
        Akteure         &  \\
        Vorbedingungen  &  \\
        Ablauf          & \begin{enumerate}
            \item 
        \end{enumerate} \\\bottomrule
    \end{tabularx}    
\end{dhbwtable}

\begin{dhbwtable}{caption={Anwendungsfall: },label=tab:use_case_,float=H}
    \begin{tabularx}{\textwidth}{lX}
        \toprule
        \textbf{Merkmal} & \textbf{Beschreibung}  \\ \midrule
        Beschreibung    &  \\
        Akteure         &  \\
        Vorbedingungen  &  \\
        Ablauf          & \begin{enumerate}
            \item 
        \end{enumerate} \\\bottomrule
    \end{tabularx}    
\end{dhbwtable}

\begin{dhbwtable}{caption={Anwendungsfall: },label=tab:use_case_,float=H}
    \begin{tabularx}{\textwidth}{lX}
        \toprule
        \textbf{Merkmal} & \textbf{Beschreibung}  \\ \midrule
        Beschreibung    &  \\
        Akteure         &  \\
        Vorbedingungen  &  \\
        Ablauf          & \begin{enumerate}
            \item 
        \end{enumerate} \\\bottomrule
    \end{tabularx}    
\end{dhbwtable}

\begin{dhbwtable}{caption={Anwendungsfall: },label=tab:use_case_,float=H}
    \begin{tabularx}{\textwidth}{lX}
        \toprule
        \textbf{Merkmal} & \textbf{Beschreibung}  \\ \midrule
        Beschreibung    &  \\
        Akteure         &  \\
        Vorbedingungen  &  \\
        Ablauf          & \begin{enumerate}
            \item 
        \end{enumerate} \\\bottomrule
    \end{tabularx}    
\end{dhbwtable}

\begin{dhbwtable}{caption={Anwendungsfall: },label=tab:use_case_,float=H}
    \begin{tabularx}{\textwidth}{lX}
        \toprule
        \textbf{Merkmal} & \textbf{Beschreibung}  \\ \midrule
        Beschreibung    &  \\
        Akteure         &  \\
        Vorbedingungen  &  \\
        Ablauf          & \begin{enumerate}
            \item 
        \end{enumerate} \\\bottomrule
    \end{tabularx}    
\end{dhbwtable}
