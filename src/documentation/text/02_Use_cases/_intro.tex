In diesem Kapitel werden die in \autoref{sec:requirements} gelisteten Anforderungen an das System in Form von Anwendungsfällen beschrieben. 

\begin{dhbwfigure}{caption=Anwendungsfall-Diagramm,label=fig:use_case_diagram,float=H}
    \begin{plantuml}
        @startuml

            left to right direction
            skinparam packageStyle rectangle

            :Kunde: as Customer
            :Verwalter: as Admin

            rectangle Ticketverkauf {
                Customer --- (Tickets für Veranstaltung bestellen)
                Customer --- (Budget abrufen)
                Customer --- (Daten zu bestellten Tickets abrufen)
                (Daten zu bestellten Tickets filtern) .> (Daten zu bestellten Tickets abrufen) : extends
                Customer ---- (Veranstaltungsliste abrufen)
                Customer ---- (Veranstaltungsdaten abrufen)

                (Anzahl verkaufter Tickets abrufen) --- Admin
                (Veranstaltung anlegen) --- Admin
                (Veranstaltungsliste abrufen) ---- Admin
                (Veranstaltungsdaten abrufen) ---- Admin
                (Kunde anlegen) ---- Admin
            }
        @enduml
    \end{plantuml}
\end{dhbwfigure}\unskip

\autoref{fig:use_case_diagram} beinhaltet ein Anwendungsfalldiagramm.
Es dient der Beschreibung von elementaren Funktionen und \dfootcite[59]{objektorientierte_systemanalyse}{[...] stellt das Zusammenspiel der Anwendungsfälle eines Systems untereinander und mit den Akteuren dar}.
Die ermittelten Anwendungsfälle werden in den Tabellen in \autoref{app:use_cases} zusätzlich näher erläutert.
