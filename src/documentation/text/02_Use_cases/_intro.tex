Im Rahmen dieses Kapitels werden die in \autoref{sec:requirements} beschriebenen Anforderungen an das System in Form von Anwendungsfällen beschrieben. 

\begin{dhbwfigure}{caption=Anwendungsfall-Diagramm,label=fig:use_case_diagram,float=H}
    \begin{plantuml}
        @startuml

            left to right direction
            skinparam packageStyle rectangle

            :Kunde: as Customer
            :Verwalter: as Admin

            rectangle Ticketverkauf {
                Customer --- (Tickets für Veranstaltung bestellen)
                Customer --- (Budget abrufen)
                Customer --- (Daten zu bestellten Tickets abrufen)
                (Daten zu bestellten Tickets filtern) .> (Daten zu bestellten Tickets abrufen) : extends
                Customer ---- (Veranstaltungsliste abrufen)
                Customer ---- (Veranstaltungsdaten abrufen)

                (Anzahl verkaufter Tickets abrufen) --- Admin
                (Veranstaltung anlegen) --- Admin
                (Veranstaltungsliste abrufen) ---- Admin
                (Veranstaltungsdaten abrufen) ---- Admin
            }
        @enduml
    \end{plantuml}
\end{dhbwfigure}

\autoref{fig:use_case_diagram} beinhaltet ein Anwendungsfalldiagramm.
Es dient der Beschreibung von elementaren Funktionen und \dfootcite[59]{objektorientierte_systemanalyse}{[...] stellt das Zusammenspiel der Anwendungsfälle eines Systems untereinander und mit den Akteuren dar}.

Die ermittelten Anwendungsfälle werden in den Tabellen \ref{tab:use_case_list_events} bis \ref{tab:use_case_add_event} genauer erläutert. % TODO

\begin{dhbwtable}{caption={Anwendungsfall: Veranstaltungsliste abrufen},label=tab:use_case_list_events,float=H}
    \begin{tabularx}{\textwidth}{lX}
        \toprule
        \textbf{Merkmal} & \textbf{Beschreibung}  \\ \midrule
        Beschreibung    & Der Akteur ruft eine Liste mit den Namen und IDs der angebotenen Veranstaltungen ab. \\
        Akteure         & Kunde, Verwalter \\
        Vorbedingungen  & Keine \\
        Ablauf          & \begin{minipage}[t]{\linewidth}
        \vspace*{-3mm}
        \begin{enumerate}[leftmargin=*,nosep,after=\strut]
            \item Der Akteur stellt über den Client eine Anfrage an die Verwaltungskomponente.
            \item Der Client gibt eine Liste mit den Daten der Veranstaltungen aus.
        \end{enumerate} 
        \end{minipage} \\\bottomrule
    \end{tabularx}    
\end{dhbwtable}

\begin{dhbwtable}{caption={Anwendungsfall: Veranstaltungsdaten abrufen},label=tab:use_case_get_event_data,float=H}
    \begin{tabularx}{\textwidth}{lX}
        \toprule
        \textbf{Merkmal} & \textbf{Beschreibung}  \\ \midrule
        Beschreibung    & Der Akteur ruft die Daten zu einer bestimmten Veranstaltung ab. \\
        Akteure         & Kunde, Verwalter \\
        Vorbedingungen  & Keine \\
        Ablauf          & \begin{minipage}[t]{\linewidth}
        \vspace*{-3mm}
        \begin{enumerate}[leftmargin=*,nosep,after=\strut]
            \item Der Akteur stellt über den Client eine Anfrage an die Verwaltungskomponente.
            \item Der Client gibt entweder die Daten zu der ausgewählten Veranstaltung aus oder die Information, dass die ausgewählte Veranstaltung nicht existiert.
        \end{enumerate}
        \end{minipage} \\\bottomrule
    \end{tabularx}    
\end{dhbwtable}

\begin{dhbwtable}{caption={Anwendungsfall: Tickets für Veranstaltung bestellen},label=tab:use_case_purchase_tickets,float=H}
    \begin{tabularx}{\textwidth}{lX}
        \toprule
        \textbf{Merkmal} & \textbf{Beschreibung}  \\ \midrule
        Beschreibung    &  Die Akteur kauft eine bestimmte Anzahl an Tickets für eine Veranstaltung. \\
        Akteure         &  Kunde \\
        Vorbedingungen  &  \begin{minipage}[t]{\linewidth}
        \vspace*{-3mm}
        \begin{itemize}[leftmargin=*,nosep,after=\strut]
            \item Die ausgewählte Veranstaltung existiert.
            \item Es sind noch Tickets verfügbar.
            \item Der Verkaufzeitraum hat bereits begonnen und ist noch nicht beendet.
            \item Das Budget des Kunden genügt für den Erwerb.
        \end{itemize}
        \end{minipage} \\
        Ablauf          & \begin{minipage}[t]{\linewidth}
        \vspace*{-3mm}
        \begin{enumerate}[leftmargin=*,nosep,after=\strut]
            \item Der Akteur stellt über den Client eine Anfrage an die Verwaltungskomponente.
            \item Die Verwaltungskomponente überprüft die Bedingungen und führt, falls möglich, die Transaktion durch.
            \item Der Client bestätigt dem Akteur mittels Bildschirmausgabe den Kauf oder teilt ihm im Fehlerfall den Grund für den fehlgeschlagenen Kauf mit.
        \end{enumerate}
        \end{minipage} \\\bottomrule
    \end{tabularx}    
\end{dhbwtable}

\begin{dhbwtable}{caption={Anwendungsfall: Budget abrufen},label=tab:use_case_get_budget,float=H}
    \begin{tabularx}{\textwidth}{lX}
        \toprule
        \textbf{Merkmal} & \textbf{Beschreibung}  \\ \midrule
        Beschreibung    &  Der Akteur ruft den aktuellen Stand seines Budgets ab. \\
        Akteure         &  Kunde \\
        Vorbedingungen  &  Keine \\
        Ablauf          & \begin{minipage}[t]{\linewidth}
        \vspace*{-3mm}
        \begin{enumerate}[leftmargin=*,nosep,after=\strut]
            \item Der Akteur stellt über den Client eine Anfrage an die Verwaltungskomponente.
            \item Der Client gibt den aktuellen Stand des Budgets aus.
        \end{enumerate}
        \end{minipage} \\\bottomrule
    \end{tabularx}    
\end{dhbwtable}

\begin{dhbwtable}{caption={Anwendungsfall: Daten zu bestellten Tickets abrufen},label=tab:use_case_list_customer_tickets,float=H}
    \begin{tabularx}{\textwidth}{lX}
        \toprule
        \textbf{Merkmal} & \textbf{Beschreibung}  \\ \midrule
        Beschreibung    & Der Akteur ruft die Daten zu seinen bestellten Tickets ab. Die Daten können hierbei nach Bestelldatum oder Zeitpunkt der Veranstaltung gefiltert werden. \\
        Akteure         & Kunde \\
        Vorbedingungen  & Keine \\
        Ablauf          & \begin{minipage}[t]{\linewidth}
        \vspace*{-3mm}
        \begin{enumerate}[leftmargin=*,nosep,after=\strut]
            \item Der Akteur stellt über den Client eine Anfrage an die Verwaltungskomponente.
            \item Der Client gibt eine Liste mit den Daten der bestellten Tickets aus.
        \end{enumerate}
        \end{minipage} \\\bottomrule
    \end{tabularx}    
\end{dhbwtable}

\begin{dhbwtable}{caption={Anwendungsfall: Anzahl verkaufter Tickets abrufen},label=tab:use_case_get_number_of_tickets_per_event,float=H}
    \begin{tabularx}{\textwidth}{lX}
        \toprule
        \textbf{Merkmal} & \textbf{Beschreibung}  \\ \midrule
        Beschreibung    & Der Akteur ruft die Anzahl der verkauften Tickets pro Veranstaltung ab. \\
        Akteure         & Verwalter \\
        Vorbedingungen  & Keine \\
        Ablauf          & \begin{minipage}[t]{\linewidth}
        \vspace*{-3mm}
        \begin{enumerate}[leftmargin=*,nosep,after=\strut]
            \item Der Akteur stellt über den Client eine Anfrage an die Verwaltungskomponente.
            \item Der Client gibt eine Liste mit der Anzahl der verkauften Tickets pro Veranstaltung aus.
        \end{enumerate}
        \end{minipage} \\\bottomrule
    \end{tabularx}    
\end{dhbwtable}

\begin{dhbwtable}{caption={Anwendungsfall: Veranstaltung anlegen},label=tab:use_case_add_event,float=H}
    \begin{tabularx}{\textwidth}{lX}
        \toprule
        \textbf{Merkmal} & \textbf{Beschreibung}  \\ \midrule
        Beschreibung    & Der Akteur legt eine Veranstaltung an. \\
        Akteure         & Verwalter \\
        Vorbedingungen  & Keine \\
        Ablauf          & \begin{minipage}[t]{\linewidth}
        \vspace*{-3mm}
        \begin{enumerate}[leftmargin=*,nosep,after=\strut]
            \item Der Akteur stellt über den Client eine Anfrage an die Verwaltungskomponente.
            \item Der Client prüft, ob alle erforderlichen Informationen übergeben wurden.
            \item Der Client gibt entweder die ID der angelegten Veranstaltung oder einen Hinweis auf fehlende Eingaben aus.
        \end{enumerate}
        \end{minipage} \\\bottomrule
    \end{tabularx}    
\end{dhbwtable}
